\documentclass[oneside,10pt,a4paper]{report}
\usepackage[a4paper, left=3cm, right=3cm, top=3cm, bottom=3cm, headsep=10mm, footskip=12mm]{geometry}
\usepackage[T1]{fontenc}
\usepackage[ngerman, english]{babel}    % mehrsprachiger Textsatz
% babel: letzte Sprache in Optionen zeigt die Sprache des Dokumentes
% und kann durch den Befehl \selectlanguage{} geaendert werden
% Passen Sie die Optionen des babel-Paketes nach Bedarf an!
\usepackage{float}
\usepackage{graphicx}
\usepackage{url}
\usepackage{pdflscape}
\usepackage{mathtools}
\usepackage{amssymb, amsmath, amstext}
\usepackage{amsthm}
\usepackage{xcolor}
\usepackage{nameref}
\usepackage{siunitx}
\usepackage{makecell}
\usepackage{hyperref}
\usepackage{enumitem}
\usepackage[superscript,biblabel]{cite}
\usepackage{caption}
\usepackage{subcaption}
\usepackage{tabularx} 			% Tabellen erzeugen
\usepackage{multirow}			 % Zeilen in Tabellenbearbeitung
\usepackage{multicol} 			% Spalten in Tabellenbearbeitung 
\usepackage{lmodern}                        % Ersatz fuer Computer Modern-Schriften 
\usepackage{amsmath}                                           % zum besseren Aussehen am Bildschirm
\usepackage{booktabs} % für schönere Tabellen
\usepackage{sidecap}
\usepackage{rotating} % für die Landscape-Umgebung
\usepackage{afterpage}
\definecolor{Bluetitle}{HTML}{1F3864}
\definecolor{softbluetitle}{HTML}{274D7E}
\definecolor{Greyish}{HTML}{5A5A5A}
%\renewcommand{\refname}{Reference}
\usepackage{array,multirow}
\newcommand{\specialcell}[2][c]{%
	\begin{tabular}[#1]{@{}c@{}}#2\end{tabular}}
\usepackage{titlesec}

\titleformat{\chapter}[display]
{\normalfont\bfseries}{}{0pt}{\Huge}

\usepackage{lipsum} 


\begin{document}
	
	\begin{titlepage}
		\begin{center}
			\begin{figure}[h!tbp]
				\includegraphics[width=\linewidth]{HUlogo.PNG}
			\end{figure}
			\vspace*{2 cm}
			
			\textcolor{Bluetitle}{\textbf{\huge Parasitologie - Praktikum}}\par
			
			\vspace*{2cm}
			\textcolor{Greyish}{\textbf{Versuchsdurchführende}}\par
			\textcolor{Greyish}{Huyen Anh Nguyen (572309)}\par

			\vspace*{0.5cm}
			\textcolor{Greyish}{\textbf{Versuchsort}}\par
			\textcolor{Greyish}{Haus 14, Kursraum}\par
			\textcolor{Greyish}{Gruppe 4}\par

			
			\vspace*{2 cm}
			\textcolor{Greyish}{\textbf{Versuchsleiter}}\par
			\textcolor{Greyish}{Prof. Dr. Kai Matuschewski}\par
			\textcolor{Greyish}{\textbf{Versuchsbetreuer}}\par
			\textcolor{Greyish}{Dr. Alexander Maier (ANU)}\par
			\textcolor{Greyish}{Linnea Polito}\par
			\textcolor{Greyish}{Grit Meusel}\par
			\textcolor{Greyish}{Dr. Richard Lucius}\par
			\textcolor{Greyish}{Peer Martin}\par
			\textcolor{Greyish}{Manuel Rauch}\par
			\textcolor{Greyish}{Dr. Katja Müller}\par
			
			
			\textcolor{Greyish}{Abgabe 15. August 2024}\par
			
			
			
		\end{center}
	\end{titlepage}

	
	\tableofcontents
	\chapter{Note from the Author}
	Das Protokoll ist eine Collection von Einzelprotokolle der Versuchen, die in den Parasitologiepraktikum durchgeführt wurde.\\
	Es wurden Methoden gezeigt, wie eine Spezies identifiziert, untersucht und dessen Verhalten analysiert werden kann.\\
	Aufgrund dieser Diversitäten an Versuchen, wird das Protokoll in vier Großkapiteln unterteilt.
	
	
	\chapter{Genetische Speziesbestimmung mittels PCR des Hirudinea}	
	\section{Einleitung}
	
	\section{Methode}
	\textit{14. Juni 2024}\\
	\\
	Für die gDNA-Isolierung wurde das Genomic DNA tissue kit verwendet.\\
	Ein grobes Stück Egelgewebe (grob 20 mg) wurde mit einem Skalpell homogenisiert und mit 400 $\mu$L Lysispuffer und 25 $\mu$L Proteinase K für 30 Minuten bei 56°C unter Schütteln inkubiert. Dann wurden 250 $\mu$L Bindingsolution SBS zugegeben und gemischt und auf einer Zentrifugenröhrchen überführt. Diese wurde für 2 Minuten bei 10000 x g zentrifugiert und das Filtrat verworfen.\\
	Dann wurde 600 $\mu$L Waschpuffer auf die Säule pipettiert und für 1 Minute für 11000 x g zentrifugiert und dieser Schritt wurde mit 300 $\mu$L Waschpuffer wiederholt. Am Ende des wurde auf maximalen Speed für 30 Sekunden die Überreste an Lösungsmittel abzentrifugiert.\\
	Das Zentrifugenröhrchen wurde auf einem 1.5 mL Eppendorf-Röhrchen überführt und auf dem Filter 50 $\mu$L RNase freies Wasser pipettiert und mit geöffneten Deckel bei 11000 x g für 2 Minuten zentrifugiert.
	
	\textit{12. Juli 2024}\\
	\\
	
	
	\textit{14. Juni 2024}\\
	\\
	\section{Ergebnis}
	\section{Diskusion}
	
	\chapter{Dichiotomische Bestimmung der Spezies des Hirudinea}
	
	\chapter{Trematoden}
	
	
	\chapter{Plasmodium}
	
	
	
\end{document}