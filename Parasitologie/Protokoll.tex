\documentclass[oneside,10pt,a4paper]{report}
\usepackage[a4paper, left=3cm, right=3cm, top=3cm, bottom=3cm, headsep=10mm, footskip=12mm]{geometry}
\usepackage[T1]{fontenc}
\usepackage[ngerman, english]{babel}    % mehrsprachiger Textsatz
% babel: letzte Sprache in Optionen zeigt die Sprache des Dokumentes
% und kann durch den Befehl \selectlanguage{} geaendert werden
% Passen Sie die Optionen des babel-Paketes nach Bedarf an!
\usepackage{float}
\usepackage{graphicx}
\usepackage{url}
\usepackage{pdflscape}
\usepackage{mathtools}
\usepackage{amssymb, amsmath, amstext}
\usepackage{amsthm}
\usepackage{xcolor}
\usepackage{nameref}
\usepackage{siunitx}
\usepackage{makecell}
\usepackage{hyperref}
\usepackage{enumitem}
\usepackage[superscript,biblabel]{cite}
\usepackage{caption}
\usepackage{subcaption}
\usepackage{tabularx} 			% Tabellen erzeugen
\usepackage{multirow}			 % Zeilen in Tabellenbearbeitung
\usepackage{multicol} 			% Spalten in Tabellenbearbeitung 
\usepackage{lmodern}                        % Ersatz fuer Computer Modern-Schriften 
\usepackage{amsmath}                                           % zum besseren Aussehen am Bildschirm
\usepackage{booktabs} % für schönere Tabellen
\usepackage{sidecap}
\usepackage{rotating} % für die Landscape-Umgebung
\usepackage{afterpage}
\definecolor{Bluetitle}{HTML}{1F3864}
\definecolor{softbluetitle}{HTML}{274D7E}
\definecolor{Greyish}{HTML}{5A5A5A}
%\renewcommand{\refname}{Reference}
\usepackage{array,multirow}
\newcommand{\specialcell}[2][c]{%
	\begin{tabular}[#1]{@{}c@{}}#2\end{tabular}}
\usepackage{titlesec}

\titleformat{\chapter}[display]
{\normalfont\bfseries}{}{0pt}{\Huge}

\usepackage{lipsum} 


\begin{document}
	
	\begin{titlepage}
		\begin{center}
			\begin{figure}[h!tbp]
				\includegraphics[width=\linewidth]{HUlogo.PNG}
			\end{figure}
			\vspace*{2 cm}
			
			\textcolor{Bluetitle}{\textbf{\huge Parasitologie - Praktikum}}\par
			
			\vspace*{2cm}
			\textcolor{Greyish}{\textbf{Versuchsdurchführende}}\par
			\textcolor{Greyish}{Huyen Anh Nguyen (572309)}\par

			\vspace*{0.5cm}
			\textcolor{Greyish}{\textbf{Versuchsort}}\par
			\textcolor{Greyish}{Haus 14, Kursraum}\par
			\textcolor{Greyish}{Gruppe 4}\par

			
			\vspace*{2 cm}
			\textcolor{Greyish}{\textbf{Versuchsleiter}}\par
			\textcolor{Greyish}{Prof. Dr. Kai Matuschewski}\par
			\vspace*{0.5cm}
			\textcolor{Greyish}{\textbf{Versuchsbetreuer}}\par
			\textcolor{Greyish}{Dr. Alexander Maier (ANU)}\par
			\textcolor{Greyish}{Linnea Polito}\par
			\textcolor{Greyish}{Grit Meusel}\par
			\textcolor{Greyish}{Dr. Richard Lucius}\par
			\textcolor{Greyish}{Peer Martin}\par
			\textcolor{Greyish}{Manuel Rauch}\par
			\textcolor{Greyish}{Dr. Katja Müller}\par

			\vspace*{2 cm}
			\textcolor{Greyish}{Abgabe 15. August 2024}\par
			
			
			
		\end{center}
	\end{titlepage}

	
	\tableofcontents
	\chapter{Note from the Author}
	Das Protokoll ist eine Collection von Einzelprotokolle der Versuchen, die in den Parasitologiepraktikum durchgeführt wurde.\\
	Es wurden Methoden gezeigt, wie eine Spezies identifiziert, untersucht und dessen Verhalten analysiert werden kann.\\
	Aufgrund dieser Diversitäten an Versuchen, wird das Protokoll in drei Großkapiteln unterteilt. Ich fand es nach den Tierklassen zu unterteilen sinnvoll, auch wenn einige Versuche einen eigenen Kapitel verdient haben.
	
	
	\chapter{Hirudinea}	
	Hirudinea sind Egel die zu der Klasse Clitellata (Gürtelwürmer) gehören, wo auch die Regenwürmer dazuzählen. Die Arten die zu den Clitellata gehören, besitzen ein sogenanntes Clitellum (verdicktes mit Epidermis, Drüsengürtel), welches das Befruchtungsort für Eizelle und Spermazellen sind. Clitellaten sind Hermaphroditen. Um Selbstbefruchtung auszuschließen, befinden sich die Spermazellen in Spermatheken und die Gameten wandern erst bei der Kokonbildung zu den Clitellum.\\
	Unter den Hirudinea gibt es räuberische (Haemopis sanguisuga), parasitische (Hirudo verbana) und aasfressende Arten, die aquatisch, amphibisch oder auch terrestrisch leben können.
	Die räuberische Arten ernähren sich von Wirbellosen und die parasitische Arten von den Wirbellosen und Wirbeltieren, wobei letzteres nicht wirtsspezifisch sind. Sie befallen jedoch Wirbeltiere in der selben Klassen \cite{Kühkental}.
	Aufgrund der fehlenden Glidern bewegen sich die Hirudinea durch kriechend, schwimmend und raupenartig fort und nutzen dabei ihre Saugnapfen als Haftwerkzeug an Oberflächen.\\
	Mit mehr als 700 Arten unter den Hirudinea und mehr als 7400 Arten unter den Clitellaten ist es wichtig die Arten richtig zu bestimmen.  Dafür stehen mehrere Methoden zur Verfügung, dies zu tun. Eine davon ist die Dichotomische Bestimmungsmethode, welches Morphologisch das untersuchende Individuum systematisch zuordnet und die genetische Analyse mittels Polymerase Chain Reaction (abgekürzt: PCR).\\
	\\
	In den Kurs wird auch noch zusätzlich die Hirudinea mit den Oligochaeta (Wenigborster) verglichen.
	
	
	
		\section{Arten Bestimmung mittels dichotomischen Bestimmungsschlüssel}\label{Abschnitt: DichoBestim}
			\subsection{Einleitung}
				Dichotomische Bestimmung ist eine Methode ein Individuum morphologisch und systematisch einer Reich, Stamm, Klasse, Ordnung und Gattung zuzuordnen und dessen Art zu bestimmen.\\
				Dichotomie kommt aus den altgriechisch und bedeutet grob übersetzt zweifacher Schnitt \cite{wiki_dichotom}.
				Der Dichotome Bestimmungsschlüssel ist eine Tabelle oder Liste von Merkmalen, die ein Experte augelistet hatte, womit das zu untersuchte Individuum abgeglichen werden kann. Dafür stehen zwei Merkmalen pro Kategorie zu (Es gibt auch Submerkmalen unter den Hauptmerkmalen, um auch kleine Abweichungen abzudecken), wo der Nutzer entscheidet welche Merkmale zu der Spezies zutrifft\cite{dichotomer_schlüssel}. Durch die Merkmalskomibantion kann somit das Taxa bestimmt werden.
			
			\subsection{Methode}
				Unter einem Stereomikroskop wurde ein in Ethanol fixiertes Egelprobe (Länge: 3 cm) untersucht.\\
				Die Merkmale wurden anhand des Dichotomischen Bestimmungsschlüssel der Egel Deutschlandsnach Clemens Großer auf Moodle systematisch zugeordnet.\\
				Damit die Probe nicht austrocknet, wurde in regelmäßigen Abstand die das Tier mit Wasser befeuchtet.
			
			\subsection{Ergebnis}
				In Figure \ref{fig:feuchte_augen} ist zu sehen, dass diese Spezies vier Augen in der vordere Reihe besitzt. Hinter der vorderen vier Augen gibt es pro Seite noch in diagonalen Stellung zwei Augen.\\
				\\
				Für die Artbestimmung wurde folgende Entscheidung getroffen:\\
				Augenzahl: 8 $\rightarrow$ 4 Augen in 1 vorderen Reihe, je 2 hinter liegenden, schräg gestellten Reihen $\rightarrow$  Dina, Erpobdella, Trocheta $\rightarrow$ der gesamter Körper abgeplattet; alle Ringe gleichbreit $\rightarrow$  Erpobdella $\rightarrow$  abgebrochen, aufgrund der knappen Zeit.\\
				\begin{figure}[H]
					\centering
					\begin{subfigure}[b]{0.55\textwidth}
						\includegraphics[width=\textwidth]{trockene_dichotom.jpg}
						\caption{Alte Egelprobe.}
						\label{fig: trockene_augen}
					\end{subfigure}
					\hfill
					\begin{subfigure}[b]{0.37\textwidth}
						\includegraphics[width=\textwidth]{Dichotom_augen}
						\caption{Frische Egelprobe}
						\label{fig:feuchte_augen}
					\end{subfigure}
					\caption{Mikroskopische Aufnahme von den Vorderkopf der Egelproben. In Figure \ref{fig: trockene_augen} handelt sich um die Egelprobe, die hier für den Versuch untersucht wurde. Aufgrund des Zustandes der Probe wurde die die Augenlage von der Komiliton*in Melissas Probe (Figure \ref{fig:feuchte_augen}) untersucht. Beide Spezies gehören zur selben Art.}
					\label{fig: Dicho_augent}
				\end{figure}
			
			
			\subsection{Diskussion}
				Anhand des Dichotomischen Bestimmungsschlüssel wurde der vorliegende Egel zu der Gattung Erpobdella zugeordnet.
				Da der Versuch nachmittags durchgeführt wurde und die Exemplare schon vormittags verwendet wurde, war der Zustand des Egels nicht mehr optimal für eine ordentliche Bestimmung.
				Damit die richtige Art bestimmt werden kann, ist eine genetische Analyse von Vorteil. Die Verwendung des Dichotomischen Bestimmungsschlüssel muss geübt werden. Eine unerfahrende Person wird dabei oftmals eine falsche Entscheidung treffen, da viele Merkmale nicht so eindeutig sind wie in den Bestimmungsschlüssel beschrieben ist.\\
				Diese morphologische Methode kann aber mit einer DNA-Barcoding unterstützt werden, welches im nächsten Abschnitt behandelt wird.
		
		\section{Arten Bestimmung mittels DNA-Barcoding}
			\subsection{Einleitung}
				Die Art des Egels wird anhand seines Genoms bestimmt. 
				Dabei werden Markergens in der Spezies molekolar untersucht, die spezifisch für eine Art konserviert ist.\\
				Da jede Art einen einzigartigen genetischen Sequenzabfolge besitzt, eignet sich diese Methode bestens um eine Spezies zu phylogenetisch zu analysieren \cite{Folmer} \cite{Herbert}.\\
				Es setzt natürlich vorraus, dass die Sequenzabfolge irgendwo auf einer Datenbank zu der Art hinterlegt ist, um mit der zu untersuchende Spezies abzugleichen.\\
				Als Sequenzmarker wird die für das  Cytocrome C Oxidase subunit 1 Enzym (abgekürzt: COI) codierte Gen in den Mitochondrien verwendet. Mitochondrialer DNA besitzen keine Introns und wird haploid vererbt (COI wird maternal vererbt), so dass es nicht wie beim Kerngenom zwei verschiedene Varianten eines Gens existieren.\\
				Damit für die Sequenzierung genug DNA-Material zur Verfügung stehen, wird hier die Polymerase Chain Reaction (abgekürzt: PCR) verwendet.\\
				Das Verfahren ermöglicht eine schnelle und hohe Kopiezahl des gewünschten Genabschnittes zu bekommen.
				Für die PCR wird ein hitzestabiles DNA-Polymerase I-Enzym (Taq-Polymerase aus dem Thermus aquaticus-Bekterium) benötigt, da die Doppelsträngige DNA zu Einzelsträngige DNA denaturiert werden muss.
				Damit die Taq-Polymerase die Amplifikation (Vervielfältigung der Nukleinsäure) durchführen kann, beötigt diese neben den Nukleotiden auch Primer, die an beide DNA-Stränge (Codogener Strang und nicht codogener Strang) an den gewünschten Genbereich bindet. Hier wird der Forward-Primer LCO1490 und Reverse Primer HC02198 von Folmer (1994) verwendet (siehe Table \ref{tab: Primer}), welches ermöglicht im GOI-Gen ein 659 bp langes Genfragmentes zu amplifizieren.\\
				Im Anschluss wird mittels einer Agarosegelelektrophorese die Fragmente nach der Größe aufgetrennt und sequenziert.\\
				
			
			\subsection{Methode}
				\textit{14. Juni 2024}\\
				\\
				Für die gDNA-Isolierung wurde das Genomic DNA tissue kit verwendet.\\
				Ein grobes Stück Egelgewebe (grob 20 mg) wurde mit einem Skalpell homogenisiert und mit 400 $\mu$L Lysispuffer und 25 $\mu$L Proteinase K für 30 Minuten bei 56°C unter Schütteln inkubiert. Dann wurden 250 $\mu$L Bindingsolution SBS zugegeben und gemischt und auf einer Zentrifugenröhrchen überführt. Diese wurde für 2 Minuten bei 10000 x g zentrifugiert und das Filtrat verworfen.\\
				Dann wurde 600 $\mu$L Waschpuffer auf die Säule pipettiert und für 1 Minute für 11000 x g zentrifugiert und dieser Schritt wurde mit 300 $\mu$L Waschpuffer wiederholt. Am Ende des wurde auf maximalen Speed für 30 Sekunden die Überreste an Lösungsmittel abzentrifugiert.\\
				Das Zentrifugenröhrchen wurde auf einem 1.5 mL Eppendorf-Röhrchen überführt und auf dem Filter 50 $\mu$L RNase freies Wasser pipettiert und mit geöffneten Deckel bei 11000 x g für 2 Minuten zentrifugiert.
				\\
				\textit{12. Juli 2024}\\
				\\
				Die Konzentration vom gDNA wurde photometrische bestimmt und beträgt 15.2 mg/mL (Wurde von Grit Meusel durchgeführt).
				Es wurden 2 $\mu$L gDNA und als negativ Kontrolle 2 $\mu$L Wasser mit 23 $\mu$L Mastermix (Pipettierschema für den Mastermix siehe Table \ref{tab: Mastermix-Pipettierschema}).
				Die Proben werden dann für 34 Cyclen amplifiziert (PCR-Cyclen siehe Table \ref{tab: PCR-Cyclen}).
				\\
				\textit{14. Juni 2024}\\
				\\
				5 $\mu$L amplfizierte PCR-Proben wird mit 1 $\mu$L 6x Gel-Puffer vermischt und auf einem 1$\%$igen Agarosegel aufgetragen. Zusätzlich wird ein DNA Marker mitraufpipettiert für die Größenbestimmung.
				Der Gellauf lief bei 100V für 1 Stunde und wurde unter ultravioletes Licht analysiert. Die Sequencebestimmung wird von den Versuchsbetreuer durchgeführt.\\
				\\
				Die Sequenzabfolge wird mit der National Library of Medicine-Datenbank abgeglichen und ein Alignment-Tree in MegaX erstellt.
				
			
			\subsection{Ergebnis}
				In Figure \ref{fig:Agarosegel} ist zu erkennen, dass es bei der Probe eine Bande existiert. Die Amplifikation hat DNA-Fragmente produziert, die bei circa 600 bp liegt. Die Negativ Kontrolle zeigt keine Bande.
				
				\begin{figure}[H]
					\centering
					\includegraphics[scale=0.7]{gel.png}
					\caption{Größenauftrennung der gDNA PCR-Probe von der unbekannte Egelgewebestücks und der Negativkontrolle (Wasser) aufgetrennt auf einer 1$\%$-Agarosegel mit einem DNA-Marker als Größenvergleich.}
					\label{fig:Agarosegel}
				\end{figure}
				
				Die Sequenz von der Isolierte Bande aus Figure \ref{fig:Agarosegel} ist in Table \ref{tab: Consensus-Sequenz Gruppen 1-7} dargestellt und glich mit 99.49 $\%$ mit der Art Erpobdella octoculata zusammen. Die Referenzsequenz stimmt mit der Art Erpobdella testacea mit der Datenbank zu 99.69 $\%$ überein.\\
				Bei der MUSCLE Alignment Algorithmus zeigen das beide Sequenzen 266 gemeinsame Basenpaare besitzen. \\
				In Figure \ref{fig: Baum} wurden die beiden Arten mit 9 weiteren Arten in Relation aufgestellt und mit der Maximum Parsimony Algorithmus den Verwandschaftsgrad bestimmt.
				
				\begin{figure}[H]
					\centering
					\includegraphics[scale=0.7]{Anhs Tree.png}
					\caption{Aligmenttree mit 10 Arten. Es wurde nach dem MUSCLE-Algorithmus alignt und ein Baum nach dem Maximium Parsimony Methode erstellt. Die Spezies die in den Kurs untersucht wurde trifft mit 99.49 $\%$ dem Erpobdella octoculata zu. Die Verwandschaftsbeziehung zu Erpobdella testacea, welches die Referenzsequenz vom Betreuer zur Verfügung stand ist in diesen Figure dargestellt. Die Rohsequenzen von der DNA-Sequenzierung ist in Table \ref{tab: Consensus-Sequenz Gruppen 1-7}}
					\label{fig: Baum}
				\end{figure}
				
			
			\subsection{Diskussion}
				Es wurde ein Sequenz amplifiziert, dass 664 bp lang ist (siehe Table \ref{tab: Consensus-Sequenz Gruppen 1-7}), welches in der Agarosegelelektrophorese auch in den Bandenbereich von circa 600 bp aufgetrennt wurde.\\
				Der Sequenzvergleich mit der Datenbank der National Library of Medicine wurde der im Versuch zu untersuchten Egel der Art Eropbdella octoculata mit einer Übereinstimmung von 99.49 $\%$.  
				In Abschnitt \ref{Abschnitt: DichoBestim} wurde anhand des Dichotomischen Bestimmungsschlüssel morphologisch die Gattung Erpobdella von den Egel bestimmt und mit der DNA-Barcoding wurde diese Gattung bestätigt.
				Der bestimmte Egelart Eropbdella octuculata und Erpobdella testacea bilden keine monophyletische Gruppe, wenn andere Egelarten in Bezug genommen wird (siehe Figure \ref{fig: Baum}) und mittels MUSCLE Alignment- und Maximum Parsimony-Algorithmus analysiert wird.\\
				Die Artenbestimmung mit der DNA-Barcoding ist deutlich kostspieliger und zeitaufwendiger als die Methode mit dem dichiotomischen Bestimmungsschlüssel. Mit beiden Methoden konnte die gleiche Gattung bestimmt werden.
				Die Sicherheit des Ergebnis ist bei beiden Methoden Abhängig von den gesammelten Daten beziehung der vorliegende Daten.\\
				
		\section{Vergleich Hirudinea und Regenwurm}
			\subsection{Einleitung}
				In der Gruppe der Clitellata gehören die Oligochaeta (Wenigborster) und die Hirudinea (Egel).
				Was diese Gruppe apomorphisch von den anderen Gruppen der Ordnung der Annaliden unterscheidet ist, dass sie keine Parapodien besitzen und Hermaphroditen sind.\\
				
			\subsection{Methode}
				Der Hirudo verbana - Egel wurde unter einem Stereo-Mikroskop präperiert und auf einem mit Alufoliebedeckten Styroformplatte fixiert. Entlang der Mittelline der Bauchseite wurde die Bauchseite mit der stumpfen Seite der Schere aufgeschnitten und aufgefächert fixiert. Der Verdauungstrakt wurde betrachtet und vorsichtig abgenommen um den inneren Rückenbereich mit den Geschlechtsorganen und Harnsystem zu betrachten. Das Nervensystem konnte nicht erkannt werden.\\
				Die Morphologie wurde dann mit den Regenwurm aus MB5 (Lumbricus terrestris) verglichen.
			\subsection{Ergebnis}
				\begin{table}[H]
					\centering
					\caption{Unterschiede in den Organsysteme zwischen der Hirudo verbana (Egel) und Lumbricus terrestris (Regenwurm).}
					\label{tab: Egel vs Regenwurm}
					\begin{tabular}{cc}
						\toprule
						\multirow{2}{*}{Verdauungssystem} & \\
						&\\
						\midrule
						\multirow{2}{*}{Kreislaufssystem}& \\	
						&\\
						\midrule
						\multirow{2}{*}{Nervensystem} & \\
						&\\
						\midrule
						\multirow{2}{*}{Fortpflanzungsorgane} & \\
						&\\
						\midrule
						\multirow{2}{*}{Harnsystem} & \\
						&\\			
						\bottomrule
					\end{tabular}
				\end{table}
			
				\begin{figure}[H]
					\centering
					\includegraphics[scale=0.25]{Egel.JPG}
					\caption{Anatomie eines Medizinischen Egels (Hirudo verbana)}
					\label{fig:Egel_ana}
				\end{figure}
				
				\begin{figure}[H]
					\centering
					\includegraphics[scale=0.25]{Regenwurm.JPG}
					\caption{Anatomie eines Regenwurmes (Limbricus terrestris)}
					\label{fig:Regenwurm_ana}
				\end{figure}
			
			\subsection{Diskussion}
			
	
	
	\chapter{Trematoden}
		\section{Malzacher-Färbung von Dicrocoelium dendriticum}
			\subsection{Einleitung}
			\subsection{Methode}
			\subsection{Ergebnis}
			\subsection{Diskussion}
		
		\section{Trematodenlarven aus Wasserschnecken}
			\subsection{Einleitung}
			\subsection{Methode}
			\subsection{Ergebnis}
			\subsection{Diskussion}
		
	
	\chapter{Plasmodium}
		\section{Giemsa-Färbung von infizierten Humanblutes}
			\subsection{Einleitung}
			\subsection{Methode}
			\subsection{Ergebnis}
			\subsection{Diskussion}
	
		\section{Darminhaltes des Blaberus craniifer}
			\subsection{Einleitung}
			\subsection{Methode}
			\subsection{Ergebnis}
			\subsection{Diskussion}
		
		\section{Cytochalasin D Wirkung auf die Motilität von Gregarinen}
			\subsection{Einleitung}
			\subsection{Methode}
			\subsection{Ergebnis}
			\subsection{Diskussion}
			
	
	
	\chapter{Anhang}
		\section{PCR}
		\begin{table}[H]
			\centering
			\caption{MasterMix Pipettierschema für 3 PCR-Raktionen. Der MasterMix ist 3-Fach konzentriert}
			\label{tab: Mastermix-Pipettierschema}
			\begin{tabular}{cc}
				\toprule
				& Volumen in $\mu$L\\
				\midrule
				10x Reaktionspuffer & 7.5 \\
				Oligo for & 1.5\\
				Oligo rev & 1.5\\
				dNTP Mix & 1.5\\
				Taq Polymerase & 0.3\\
				H$_2$O & 57\\
				\bottomrule
			\end{tabular}
		\end{table}
		
		
		\begin{table}[H]
			\centering
			\caption{PCR-Cyclus Einstellungsprogramm. 34 Cyclen wurde durchgeführt.}
			\label{tab: PCR-Cyclen}
			\begin{tabular}{cc}
				\toprule
				94°C & 3 Minute\\
				94°C & 45 Sekunde\\
				48°C & 45 Sekunde\\
				72°C & 1 Minute\\
				72°C & 5 Minute\\
				4°C & unendlich\\
				\bottomrule
			\end{tabular}
		\end{table}
		
		\begin{table}[H]
			\centering
			\caption{Sequenz vom Forward- und Reverse-Primer für die PCR. Folmer (1994).}
			\label{tab: Primer}
			\begin{tabular}{cc}
				Forward & 5’ GGTCAACAAATCATAAAGATATTGG 3'\\
				Reverse & 5’ TAAACTTCAGGGTGACCAAAAAATCA 3’\\				
			\end{tabular}
		\end{table}
		
		\begin{table}[H]
			\centering
			\caption{Consensus-Sequenz Gruppen 1-7. Die Bande im Bereich 600 bp wurde aus der Agarosegelelektrophorese isoliert und sequenziert.}
			\label{tab: Consensus-Sequenz Gruppen 1-7}
			\begin{tabular}{l}
				TTTATTCTAGGAAGCATGGGTCAGCTATAGCTGGCACAGGCATAAGGGTACTAATTCGAA\\
				TTGAGTTAGCCCAACCTGGCACATTTTTAGGAAACGATCAAATTTATAACACTATTGTAA\\
				CCGCTCACGGGCTAGTAATAATTTTCTTTATAGTAATACCTATTTTAATTGGAGGATTTG\\
				GAAATTGGTTGATTCCATTAATAATTGGTGCACCAGATATAGCCTTTCCTCGACTCAATA\\
				ATCTAAGATTTTGACTATTACCCCCATCAATAATTATATTAGTCTTCTCTGCATTTGTAG\\
				AAAATGGTGTGGGTACTGGATGAACAGTATACCCTCCCTTAGCATATAATATTGCCCACT\\
				CTGGCCCATCAGTAGATATGGCTATTTTCTCATTACATTTAGCAGGAGCTTCATCTATTT\\
				TAGGATCATTAAACTTTATTTCCACTGTAGCAAATATACGATGAAAAGGTATATCATTAG\\
				ATCGAATCCCTTTATTTATTTGATCAGTAATTATTACTACAGTACTTCTACTTCTATCAT\\
				TACCAGTTTTAGCAGCTGCCATTACTATATTACTGACTGATCGAAACTTAAATACATCCT\\
				TCTTTGACCCTGCAGGAGGAGGAGACCCAATCCTATTCCAACATTTATTTTGATTTTTTG\\
				GTCACC\\
			\end{tabular}
		\end{table}
		
		\begin{table}[H]
			\centering
			\caption{Consencus-Referenzsequenz vom anderen morphologischen Tier vom Betreuer zur Verfügung gestellt.}
			\label{tab: Vergleichssequenz}
			\begin{tabular}{l}
				TTTTAGGGGCCTGATCAGCTATAGCTGGTACAGGCATAAGAGTACTAATTCGAATTGAAT\\
				TAGCTCAACCCGGCACATTTCTAGGAAATGATCAAATTTATAATACTATTGTAACCGCAC\\
				ATGGATTAGTAATAATCTTCTTTATAGTAATACCTATTTTAATTGGAGGATTTGGTAATT\\
				GGTTAATTCCATTAATAATTGGGGCCCCAGATATAGCATTTCCGCGACTTAATAATCTAA\\
				GATTCTGACTACTTCCACCATCCATAATTATACTAGTATTCTCTGCATTTGTAGAAAATG\\
				GAGTAGGTACTGGATGAACAGTATACCCTCCTCTAGCATACAATATTGCACACTCTGGCC\\
				CATCTGTAGATATAGCTATCTTCTCATTACATTTAGCGGGAGCATCATCTATCCTAGGGT\\
				CCCTAAATTTTATTTCAACTGTAGCTAATATACGATGAAAAGGAATAACTATAGATCGAA\\
				TTCCTTTATTTATTTGATCAGTAATTATTACTACAGTTCTTTTACTACTCTCCTTACCTG\\
				TACTAGCAGCAGCAATTACAATATTATTAACTGATCGAAACCTAAATACATCATTCTTTG\\
				ACCCAGCAGGTGGAGGAGATCCTATTTTATTCCAACACTTATTTTGATTTTTTG\\
				
			\end{tabular}
		\end{table}
		
		
		
		
		
		
		
	\addcontentsline{toc}{section}{Bibliography}
	\bibliographystyle{plainurl}
	\nocite{*}
	\bibliography{Literatur}
	\newpage
\end{document}