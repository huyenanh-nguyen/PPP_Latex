\documentclass[oneside,10pt,a4paper]{report}
\usepackage[a4paper, left=3cm, right=3cm, top=3cm, bottom=3cm, headsep=10mm, footskip=12mm]{geometry}
\usepackage[T1]{fontenc}
\usepackage[ngerman, english]{babel}    % mehrsprachiger Textsatz
% babel: letzte Sprache in Optionen zeigt die Sprache des Dokumentes
% und kann durch den Befehl \selectlanguage{} geaendert werden
% Passen Sie die Optionen des babel-Paketes nach Bedarf an!
\usepackage{float}
\usepackage{graphicx}
\usepackage{url}
\usepackage{pdflscape}
\usepackage{mathtools}
\usepackage{amssymb, amsmath, amstext}
\usepackage{amsthm}
\usepackage{xcolor}
\usepackage{nameref}
\usepackage{siunitx}
\usepackage{makecell}
\usepackage{hyperref}
\usepackage{enumitem}
\usepackage[superscript,biblabel]{cite}
\usepackage{caption}
\usepackage{subcaption}
\usepackage{tabularx} 			% Tabellen erzeugen
\usepackage{multirow}			 % Zeilen in Tabellenbearbeitung
\usepackage{multicol} 			% Spalten in Tabellenbearbeitung 
\usepackage{lmodern}                        % Ersatz fuer Computer Modern-Schriften 
\usepackage{amsmath}                                           % zum besseren Aussehen am Bildschirm
\usepackage{booktabs} % für schönere Tabellen
\usepackage{sidecap}
\usepackage{rotating} % für die Landscape-Umgebung
\usepackage{afterpage}
\definecolor{Bluetitle}{HTML}{1F3864}
\definecolor{softbluetitle}{HTML}{274D7E}
\definecolor{Greyish}{HTML}{5A5A5A}
%\renewcommand{\refname}{Reference}
\usepackage{array,multirow}
\newcommand{\specialcell}[2][c]{%
	\begin{tabular}[#1]{@{}c@{}}#2\end{tabular}}
\usepackage{titlesec}

\titleformat{\chapter}[display]
{\normalfont\bfseries}{}{0pt}{\Huge}

\usepackage{lipsum} 


\begin{document}
	
	\begin{titlepage}
		\begin{center}
			\begin{figure}[h!tbp]
				\includegraphics[width=\linewidth]{HUlogo.PNG}
			\end{figure}
			\vspace*{2 cm}
			
			\textcolor{Bluetitle}{\textbf{\huge Parasitologie - Praktikum}}\par
			
			\vspace*{2cm}
			\textcolor{Greyish}{\textbf{Versuchsdurchführende}}\par
			\textcolor{Greyish}{Huyen Anh Nguyen (572309)}\par

			\vspace*{0.5cm}
			\textcolor{Greyish}{\textbf{Versuchsort}}\par
			\textcolor{Greyish}{Haus 14, Kursraum}\par
			\textcolor{Greyish}{Gruppe 4}\par

			
			\vspace*{2 cm}
			\textcolor{Greyish}{\textbf{Versuchsleiter}}\par
			\textcolor{Greyish}{Prof. Dr. Kai Matuschewski}\par
			\textcolor{Greyish}{\textbf{Versuchsbetreuer}}\par
			\textcolor{Greyish}{Dr. Alexander Maier (ANU)}\par
			\textcolor{Greyish}{Linnea Polito}\par
			\textcolor{Greyish}{Grit Meusel}\par
			\textcolor{Greyish}{Dr. Richard Lucius}\par
			\textcolor{Greyish}{Peer Martin}\par
			\textcolor{Greyish}{Manuel Rauch}\par
			\textcolor{Greyish}{Dr. Katja Müller}\par
			
			
			\textcolor{Greyish}{Abgabe 15. August 2024}\par
			
			
			
		\end{center}
	\end{titlepage}

	
	\tableofcontents
	\chapter{Note from the Author}
	Das Protkoll st eine Collection von Einzelprotokolle der Versuchen, die in den Parasitologiepraktikum durchgeführt wurde.\\
	Es wurden Methoden gezeigt, wie eine Spezies identifiziert, untersucht und dessen Verhalten analysiert werden kann.\\
	
	
	In diesen Parasitologie-Praktikum wurden 4 verschiedene Methoden für die Quantifizierung und Qualifizierung von verschiedenen Parasitenarten vorgestellt.\\
	Das Protokoll wird in 4 Hauptkapiteln unterteilt, die jeweils für sich individuell betrachtet werden,
	\chapter{Genetische Speziesbestimmung mittels PCR des Hirudinea}	
	\section{Einleitung}
	
	\section{Methode}
	\section{Ergebnis}
	\section{Diskusion}
	
	\chapter{Dichiotomische Bestimmung der Spezies des Hirudinea}
	
	\chapter{Trematoden}
	
	
	\chapter{Plasmodium}
	
	
	
\end{document}