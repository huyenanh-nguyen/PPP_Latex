\documentclass[oneside,10pt,a4paper]{report}
\usepackage[a4paper, left=3cm, right=3cm, top=3cm, bottom=3cm, headsep=10mm, footskip=12mm]{geometry}
\usepackage[T1]{fontenc}
\usepackage[ngerman, english]{babel}    % mehrsprachiger Textsatz
% babel: letzte Sprache in Optionen zeigt die Sprache des Dokumentes
% und kann durch den Befehl \selectlanguage{} geaendert werden
% Passen Sie die Optionen des babel-Paketes nach Bedarf an!
\usepackage{float}
\usepackage{graphicx}
\usepackage{url}
\usepackage{pdflscape}
\usepackage{mathtools}
\usepackage{amssymb, amsmath, amstext}
\usepackage{amsthm}
\usepackage{xcolor}
\usepackage{nameref}
\usepackage{siunitx}
\usepackage{makecell}
\usepackage{hyperref}
\usepackage{enumitem}
\usepackage[superscript,biblabel]{cite}
\usepackage{caption}
\usepackage{subcaption}
\usepackage{tabularx} 			% Tabellen erzeugen
\usepackage{multirow}			 % Zeilen in Tabellenbearbeitung
\usepackage{multicol} 			% Spalten in Tabellenbearbeitung 
\usepackage{lmodern}                        % Ersatz fuer Computer Modern-Schriften 
\usepackage{amsmath}                                           % zum besseren Aussehen am Bildschirm
\usepackage{booktabs} % für schönere Tabellen
\usepackage{sidecap}
\usepackage{rotating} % für die Landscape-Umgebung
\usepackage{afterpage}
\definecolor{Bluetitle}{HTML}{1F3864}
\definecolor{softbluetitle}{HTML}{274D7E}
\definecolor{Greyish}{HTML}{5A5A5A}
%\renewcommand{\refname}{Reference}
\usepackage{array,multirow}
\newcommand{\specialcell}[2][c]{%
	\begin{tabular}[#1]{@{}c@{}}#2\end{tabular}}
\usepackage{titlesec}

\titleformat{\chapter}[display]
{\normalfont\bfseries}{}{0pt}{\Huge}

\usepackage{lipsum} 


\begin{document}
	
	\begin{titlepage}
		\begin{center}
			\begin{figure}[h!tbp]
				\includegraphics[width=\linewidth]{HUlogo.PNG}
			\end{figure}
			\vspace*{2 cm}
			
			\textcolor{Bluetitle}{\textbf{\huge Molekulare Mikrobiologie}}\par
			
			\vspace*{2cm}
			\textcolor{Greyish}{\textbf{Versuchsdurchführende}}\par
			\textcolor{Greyish}{Huyen Anh Nguyen (572309)}\par
			
			\vspace*{0.5cm}
			\textcolor{Greyish}{\textbf{Versuchsort}}\par
			\textcolor{Greyish}{Haus 14, Kursraum}\par
			\textcolor{Greyish}{Gruppe 4}\par
			
			
			\vspace*{2 cm}
			\textcolor{Greyish}{\textbf{Versuchsleiter}}\par
			\textcolor{Greyish}{Prof. Dr. Marc Erhardt}\par
			\vspace*{0.5cm}
			\textcolor{Greyish}{\textbf{Versuchsbetreuer}}\par
			\textcolor{Greyish}{Dr. Caroline Kühne}\par
			\textcolor{Greyish}{Heidi Landmesser}\par

			
			\vspace*{2 cm}
			\textcolor{Greyish}{Abgabe 31. Januar 2025}\par
			
			
			
		\end{center}
	\end{titlepage}
	
	
	\tableofcontents
	\chapter{Note from the Author}
	Ich erkläre ausdrücklich, dass ich die Anmerkungen zur Anfertigung des Protokolls gelesen und befolgt habe, dass es sich bei der von mir eingereichte Arbeit um eine von mir erstmalig, selbstständig ohne fremde Hilfe verfasste Arbeit handelt und dass ich sämtliche verwendete zulässige Literatur (Fachpublikationen/-bücher), die unverändert oder abgewandelt wiedergeben werde, insbesondere Quellen für Texte, Grafiken, Tabellen und Bilder als solche kenntlich gemacht habe.\\
	Mir ist bewusst, dass Verstöße gegen diese Grundsätze als Täuschung betrachtet und entsprechend der Prüfungsordnung und/oder der Fächerübergreifenden Satzung zur REglung von Zulassung, Studium und Prüfung der Humboldt-Universität zu Berlin geahndet werden.
	
	
	\chapter{Insertion mutagenesis using the transposable element T-Pop}	
	
		\section{Einleitung}
		
		\section{Methode}
		In Table \ref{tab: exp2-biologisches Material} wurden die in dem Versuch verwendete Biologisches Material aufgeführt
		
			\begin{table}[H]
			\centering
			\caption{Verwendete biologische Material für die Insertion von Mutagene mittels eines T-Pop Transposon.}
			\label{tab: exp2-biologisches Material}
			\begin{tabular}{ccc}
				\toprule
				Biologisches Material& Stamm & Phenotyp\\
				\midrule
				\multirow{2}{*}{P22 Phagen Lysat} & \multirow{2}{*}{TH3468} & \multirow{2}{*}{\parbox[t]{9cm}{F’128 (pro-lac) zzf-3834::Tn10dTc[del-20 del-25] (T-POP3) / proAB4}}\\
				&&\\
				&&\\
				\multirow{3}{*}{\parbox[t]{3cm}{Recipienten Salmonellazellen}} & \multirow{3}{*}{EM8052} &\multirow{3}{*}{\parbox[t]{9cm}{MvP103 sseC::aphT (KmR) fljB23028::MudJ-Cm (Km in MudJ replaced by FCF)/ pNK2880 (ApR)}} \\
				&&\\
				&&\\

				\bottomrule			
			\end{tabular}
		\end{table}
		\textit{Tag 1: 13.11.2024}
		
		\section{Ergebnis}
		\section{Diskussion}

	
	\chapter{Anhang}

	
	
	
	\addcontentsline{toc}{section}{Bibliography}
	\bibliographystyle{plainurl}
	\nocite{*}
	\bibliography{Literatur}
	\newpage
\end{document}