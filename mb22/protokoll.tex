\documentclass[oneside,10pt,a4paper]{report}
\usepackage[a4paper, left=3cm, right=3cm, top=3cm, bottom=3cm, headsep=10mm, footskip=12mm]{geometry}
\usepackage[T1]{fontenc}
\usepackage[ngerman, english]{babel}    % mehrsprachiger Textsatz
% babel: letzte Sprache in Optionen zeigt die Sprache des Dokumentes
% und kann durch den Befehl \selectlanguage{} geaendert werden
% Passen Sie die Optionen des babel-Paketes nach Bedarf an!
\usepackage{float}
\usepackage{graphicx}
\usepackage{url}
\usepackage{pdflscape}
\usepackage{mathtools}
\usepackage{amssymb, amsmath, amstext}
\usepackage{amsthm}
\usepackage{xcolor}
\usepackage{nameref}
\usepackage{siunitx}
\usepackage{makecell}
\usepackage{hyperref}
\usepackage{enumitem}
\usepackage[superscript,biblabel]{cite}
\usepackage{caption}
\usepackage{subcaption}
\usepackage{tabularx} 			% Tabellen erzeugen
\usepackage{multirow}			 % Zeilen in Tabellenbearbeitung
\usepackage{multicol} 			% Spalten in Tabellenbearbeitung 
\usepackage{lmodern}                        % Ersatz fuer Computer Modern-Schriften 
\usepackage{amsmath}                                           % zum besseren Aussehen am Bildschirm
\usepackage{booktabs} % für schönere Tabellen
\usepackage{sidecap}
\usepackage{rotating} % für die Landscape-Umgebung
\usepackage{afterpage}
\definecolor{Bluetitle}{HTML}{1F3864}
\definecolor{softbluetitle}{HTML}{274D7E}
\definecolor{Greyish}{HTML}{5A5A5A}
%\renewcommand{\refname}{Reference}
\usepackage{array,multirow}
\newcommand{\specialcell}[2][c]{%
	\begin{tabular}[#1]{@{}c@{}}#2\end{tabular}}
\usepackage{titlesec}

\titleformat{\chapter}[display]
{\normalfont\bfseries}{}{0pt}{\Huge}

\usepackage{lipsum} 


\begin{document}
	
	\begin{titlepage}
		\begin{center}
			\begin{figure}[h!tbp]
				\includegraphics[width=\linewidth]{HUlogo.PNG}
			\end{figure}
			\vspace*{2 cm}
			
			\textcolor{Bluetitle}{\textbf{\huge Molekulare Mikrobiologie}}\par
			
			\vspace*{2cm}
			\textcolor{Greyish}{\textbf{Versuchsdurchführende}}\par
			\textcolor{Greyish}{Huyen Anh Nguyen (572309)}\par
			
			\vspace*{0.5cm}
			\textcolor{Greyish}{\textbf{Versuchsort}}\par
			\textcolor{Greyish}{Haus 14, Kursraum}\par
			\textcolor{Greyish}{Gruppe 4}\par
			
			
			\vspace*{2 cm}
			\textcolor{Greyish}{\textbf{Versuchsleiter}}\par
			\textcolor{Greyish}{Prof. Dr. Marc Erhardt}\par
			\vspace*{0.5cm}
			\textcolor{Greyish}{\textbf{Versuchsbetreuer}}\par
			\textcolor{Greyish}{Dr. Caroline Kühne}\par
			\textcolor{Greyish}{Heidi Landmesser}\par

			
			\vspace*{2 cm}
			\textcolor{Greyish}{Abgabe 31. Januar 2025}\par
			
			
			
		\end{center}
	\end{titlepage}
	
	
	\tableofcontents
	\chapter{Note from the Author}
	Ich erkläre ausdrücklich, dass ich die Anmerkungen zur Anfertigung des Protokolls gelesen und befolgt habe, dass es sich bei der von mir eingereichte Arbeit um eine von mir erstmalig, selbstständig ohne fremde Hilfe verfasste Arbeit handelt und dass ich sämtliche verwendete zulässige Literatur (Fachpublikationen/-bücher), die unverändert oder abgewandelt wiedergeben werde, insbesondere Quellen für Texte, Grafiken, Tabellen und Bilder als solche kenntlich gemacht habe.\\
	Mir ist bewusst, dass Verstöße gegen diese Grundsätze als Täuschung betrachtet und entsprechend der Prüfungsordnung und/oder der Fächerübergreifenden Satzung zur REglung von Zulassung, Studium und Prüfung der Humboldt-Universität zu Berlin geahndet werden.
	
	
	\chapter{Insertion mutagenesis using the transposable element T-Pop}	
	
		\section{Einleitung}
		
		\section{Methode}
		In Table \ref{tab: exp2-biologisches Material} wurden die in dem Versuch verwendete Biologisches Material aufgeführt.
		
			\begin{table}[H]
			\centering
			\caption{Verwendete biologische Material für die Insertion von Mutagene mittels eines T-Pop Transposon.}
			\label{tab: exp2-biologisches Material}
			\begin{tabular}{ccc}
				\toprule
				Biologisches Material& Stamm & Phänotyp\\
				\midrule
				\multirow{2}{*}{P22 Phagen Lysat} & \multirow{2}{*}{TH3468} & \multirow{2}{*}{\parbox[t]{9cm}{F’128 (pro-lac) zzf-3834::Tn10dTc[del-20 del-25] (T-POP3) / proAB4}}\\
				&&\\
				&&\\
				\multirow{3}{*}{\parbox[t]{3cm}{Recipienten Salmonellazellen}} & \multirow{3}{*}{EM8052} &\multirow{3}{*}{\parbox[t]{9cm}{MvP103 sseC::aphT (KmR) fljB23028::MudJ-Cm (Km in MudJ replaced by FCF)/ pNK2880 (Ap$^R$)}} \\
				&&\\
				&&\\

				\bottomrule			
			\end{tabular}
		\end{table}

		\textit{Tag 1: 13.11.2024}\\
		Es wurde wie im Skript \cite{Mibi-Script} drei verschiedene Probe-Lösungen auf den LB-Agarplatten ausplattiert: Recipientenzellen mit dem Phagenlysat (drei Replikate), Recipientenzellen mit PBS-Puffer (Phosphatpuffer)und Phagenlysat mit PBS-Puffer.\\
		Die Platten wurden über Nacht bei 37°C inkubiert
		\\
		\textit{Tag 2: 14.11.2024}\\
		Die drei Platten mit den Recipientenzellen und Phagenlysat wurde wie im Skript \cite{Mibi-Script} auf MacConkey-lactose Platten mit Tetracyclin(abgekürzt: Mac-Lac-Tc) und MacConkey-lactose Platten ohne Tetracyclin(abgekürzt: Mac-Lac) mittels Replikationsstempel repliziert. und über Nacht bei 37°C inkubiert.\\
		\textit{Tag 3: 15.11.2024}\\
		Kolonien der jeweiligen Platten wurden ermittelt und den Phänotyp bestimmt.
		\section{Ergebnis}
			\begin{table}[H]
			\centering
			\caption{Mittelwert von der Table \ref{tab: exp2-Rohdaten}. Der Transposon T-POP wurde mit dem Phagenstamm TH3468 in die Salmonella-Recipientenstamm EM8052 in den Genom eingefügt. Hier wurde der Mittelwert der Anzahl der Gene ermittelt, die durch die Insertion betroffen sind.}
			\label{tab: exp2-ergebnis}
			\begin{tabular}{ccccccc}
				\toprule
			\multirow{2}{*}{$\#$TcR} & \multirow{2}{*}{Lac$^-$}&\multirow{2}{*}{lac$^-$ (beide Platten)} & \multirow{2}{*}{Tc-dep.-Lac$^-$}& \multirow{2}{*}{Tc-dep.Lac$^+$}& \multirow{2}{*}{Ratio: $\frac{Lac^-}{\#TcR}$}&\multirow{2}{*}{\parbox[*]{1.2cm}{Genes affected}}\\
				&&&&&&\\
				\midrule
				130 & 2 & 1.5 & 0.5 & 0 & 2/130 & 72\\
				\bottomrule			
			\end{tabular}
		\end{table}
		\section{Diskussion}

	
	\chapter{Anhang}
\begin{table}[H]
	\centering
	\caption{Der Transposon T-POP wurde in die Recipientenzellen des Stammes EM8052 (Table \ref{tab: exp2-biologisches Material}) über Nacht mit einem Phagenlysat TH3468 (Table \ref{tab: exp2-biologisches Material}) transduziert. Anzahl der Kolonien von Gruppe 1-8, die auf MacConkey-lactose Platten mit Tetracyclin und MacConkey-lactose Platten ohne Tetracyclin gewachsen sind, wurden bestimmt und die Anzahl der betroffene Gene ermittelt.}
	\label{tab: exp2-Rohdaten}
	\begin{tabular}{cccccccc}
		\toprule
		\multirow{2}{*}{Gr.} & \multirow{2}{*}{$\#$TcR} & \multirow{2}{*}{Lac$^-$}&\multirow{2}{*}{lac$^-$ (beide Platten)} & \multirow{2}{*}{Tc-dep.-Lac$^-$}& \multirow{2}{*}{Tc-dep.Lac$^+$}& \multirow{2}{*}{Ratio: $\frac{Lac^-}{\#TcR}$}&\multirow{2}{*}{\parbox[*]{1.2cm}{Genes affected}}\\
		&&&&&&&\\
		\midrule
		1 & 246 & 5 & 3 & 2 & 0 & 5/246 & 81\\
		2 & 77 & 1 & 0 & 0 & 1 & 1/77 & 52\\
		3 & 115 & 1 & 1 & 0 & 0 & 1/115& 35\\
		4 & 93 & 2 & 1 & 1 & 0 & 2/93 & 86\\
		5 & 33 & 1 & 1 & 0 & 0 & 1/33& 121\\
		6 & 95 & 0 & 0 & 0 & 0 & NA & NA\\
		7 & 106 & 1 & 1 & 0 & 0 & 1/106 & 38\\
		8 & 274 & 6 & 5 & 1 & 0 & 6/274 & 88\\
		\bottomrule			
	\end{tabular}
\end{table}
	
	
	
	\addcontentsline{toc}{section}{Bibliography}
	\bibliographystyle{plainurl}
	\nocite{*}
	\bibliography{Literatur}
	\newpage
\end{document}