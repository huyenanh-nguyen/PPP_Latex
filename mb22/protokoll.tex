\documentclass[oneside,10pt,a4paper]{report}
\usepackage[a4paper, left=3cm, right=3cm, top=3cm, bottom=3cm, headsep=10mm, footskip=12mm]{geometry}
\usepackage[T1]{fontenc}
\usepackage[ngerman, english]{babel}    % mehrsprachiger Textsatz
% babel: letzte Sprache in Optionen zeigt die Sprache des Dokumentes
% und kann durch den Befehl \selectlanguage{} geaendert werden
% Passen Sie die Optionen des babel-Paketes nach Bedarf an!
\usepackage{float}
\usepackage{graphicx}
\usepackage{url}
\usepackage{pdflscape}
\usepackage{mathtools}
\usepackage{amssymb, amsmath, amstext}
\usepackage{amsthm}
\usepackage{xcolor}
\usepackage{nameref}
\usepackage{siunitx}
\usepackage{makecell}
\usepackage{hyperref}
\usepackage{enumitem}
\usepackage[superscript,biblabel]{cite}
\usepackage{caption}
\usepackage{subcaption}
\usepackage{tabularx} 			% Tabellen erzeugen
\usepackage{multirow}			 % Zeilen in Tabellenbearbeitung
\usepackage{multicol} 			% Spalten in Tabellenbearbeitung 
\usepackage{lmodern}                        % Ersatz fuer Computer Modern-Schriften 
\usepackage{amsmath}                                           % zum besseren Aussehen am Bildschirm
\usepackage{booktabs} % für schönere Tabellen
\usepackage{sidecap}
\usepackage{rotating} % für die Landscape-Umgebung
\usepackage{afterpage}
\definecolor{Bluetitle}{HTML}{1F3864}
\definecolor{softbluetitle}{HTML}{274D7E}
\definecolor{Greyish}{HTML}{5A5A5A}
%\renewcommand{\refname}{Reference}
\usepackage{array,multirow}
\newcommand{\specialcell}[2][c]{%
	\begin{tabular}[#1]{@{}c@{}}#2\end{tabular}}
\usepackage{titlesec}

\titleformat{\chapter}[display]
{\normalfont\bfseries}{}{0pt}{\Huge}

\usepackage{lipsum} 


\begin{document}
	
	\begin{titlepage}
		\begin{center}
			\begin{figure}[h!tbp]
				\includegraphics[width=\linewidth]{HUlogo.PNG}
			\end{figure}
			\vspace*{2 cm}
			
			\textcolor{Bluetitle}{\textbf{\huge Molekulare Mikrobiologie}}\par
			
			\vspace*{2cm}
			\textcolor{Greyish}{\textbf{Versuchsdurchführende}}\par
			\textcolor{Greyish}{Huyen Anh Nguyen (572309)}\par
			
			\vspace*{0.5cm}
			\textcolor{Greyish}{\textbf{Versuchsort}}\par
			\textcolor{Greyish}{Haus 14, Kursraum}\par
			\textcolor{Greyish}{Gruppe 4}\par
			
			
			\vspace*{2 cm}
			\textcolor{Greyish}{\textbf{Versuchsleiter}}\par
			\textcolor{Greyish}{Prof. Dr. Marc Erhardt}\par
			\vspace*{0.5cm}
			\textcolor{Greyish}{\textbf{Versuchsbetreuer}}\par
			\textcolor{Greyish}{Dr. Caroline Kühne}\par
			\textcolor{Greyish}{Heidi Landmesser}\par

			
			\vspace*{2 cm}
			\textcolor{Greyish}{Abgabe 31. Januar 2025}\par
			
			
			
		\end{center}
	\end{titlepage}
	
	
	\tableofcontents
	\chapter{Note from the Author}
	Wir erkläres ausdrücklich, dass wir die Anmerkungen zur Anfertigung des Protokolls gelesen und befolgt haben, dass es sich bei der von uns eingereichte Arbeit um eine von uns erstmalig, selbstständig ohne fremde Hilfe verfasste Arbeit handelt und dass wir sämtliche verwendete zulässige Literatur (Fachpublikationen/-bücher), die unverändert oder abgewandelt wiedergeben werden, insbesondere Quellen für Texte, Grafiken, Tabellen und Bilder als solche kenntlich gemacht haben.\\
	Uns ist bewusst, dass Verstöße gegen diese Grundsätze als Täuschung betrachtet und entsprechend der Prüfungsordnung und/oder der Fächerübergreifenden Satzung zur Reglung von Zulassung, Studium und Prüfung der Humboldt-Universität zu Berlin geahndet werden.
	
	
	\chapter{Insertion mutagenesis using the transposable element T-Pop}	
	
		\section{Einleitung}
		In diesen Experiment wird untersucht, wie viele Gene des Salmonella Chromosom durch die Insertion des T-POP-Transposon in seinem Salmonella-Stamm mit einem fljB-lac Fusion beeinflusst werden.
		Für die Proteinfilamente des Flagellums sind die zwei Gene FljB und FliC (der Klasse 3- Proteine) in der Salmonella verantwortlich\cite{Flagellum}. Der Salmonella-Stanm für diesen Versuch besitzt in seinem fljB-Gen den Mud-lac-Operon, ein Fusionsgen. (MudJ besitzt ein Kanamycin-Resistenz Kassette fusioniert mit den lac-Operon, wobei hier der Kanamycin-Resistenz Kassette durch ein Chloramphenicol-Resistenz Kassette ersetzt wird.)
		Das bedeutet, wenn das fljB-Gen exprimiert wird, wird zugleich auch der lac Operon mit exprimiert, da der lac Operon den fljB Promotor mitbenutzt.
		
		
		\section{Methode}
		In Table \ref{tab: exp2-biologisches Material} wurden die in dem Versuch verwendete Biologisches Material aufgeführt.
		
			\begin{table}[H]
			\centering
			\caption{Verwendete biologische Material für die Insertion von Mutagene mittels eines T-Pop Transposon.}
			\label{tab: exp2-biologisches Material}
			\begin{tabular}{ccc}
				\toprule
				Biologisches Material& Stamm & Phänotyp\\
				\midrule
				\multirow{2}{*}{P22 Phagen Lysat} & \multirow{2}{*}{TH3468} & \multirow{2}{*}{\parbox[t]{9cm}{F’128 (pro-lac) zzf-3834::Tn10dTc[del-20 del-25] (T-POP3) / proAB4}}\\
				&&\\
				&&\\
				\multirow{3}{*}{\parbox[t]{3cm}{Recipienten Salmonellazellen}} & \multirow{3}{*}{EM8052} &\multirow{3}{*}{\parbox[t]{9cm}{MvP103 sseC::aphT (Km$^R$) fljB23028::MudJ-Cm (Km in MudJ replaced by FCF)/ pNK2880 (Ap$^R$)}} \\
				&&\\
				&&\\

				\bottomrule			
			\end{tabular}
		\end{table}

		\textit{Tag 1: 13.11.2024}\\
		Es wurde wie im Skript \cite{Mibi-Script} drei verschiedene Probe-Lösungen auf den LB-Agarplatten ausplattiert: Recipientenzellen mit dem Phagenlysat (drei Replikate), Recipientenzellen mit PBS-Puffer (Phosphatpuffer)und Phagenlysat mit PBS-Puffer.\\
		Die Platten wurden über Nacht bei 37°C inkubiert
		\\
		\textit{Tag 2: 14.11.2024}\\
		Die drei Platten mit den Recipientenzellen und Phagenlysat wurde wie im Skript \cite{Mibi-Script} auf MacConkey-lactose Platten mit Tetracyclin(abgekürzt: Mac-Lac-Tc) und MacConkey-lactose Platten ohne Tetracyclin(abgekürzt: Mac-Lac) mittels Replikationsstempel repliziert. und über Nacht bei 37°C inkubiert.\\
		\textit{Tag 3: 15.11.2024}\\
		Kolonien der jeweiligen Platten wurden ermittelt und den Phänotyp bestimmt.
		\section{Ergebnis}
			\begin{table}[H]
			\centering
			\caption{Mittelwert von der Table \ref{tab: exp2-Rohdaten}. Der Transposon T-POP wurde mit dem Phagenstamm TH3468 in die Salmonella-Recipientenstamm EM8052 in den Genom eingefügt. Hier wurde der Mittelwert der Anzahl der Gene ermittelt, die durch die Insertion betroffen sind.}
			\label{tab: exp2-ergebnis}
			\begin{tabular}{ccccccc}
				\toprule
			\multirow{2}{*}{$\#$TcR} & \multirow{2}{*}{Lac$^-$}&\multirow{2}{*}{lac$^-$ (beide Platten)} & \multirow{2}{*}{Tc-dep.-Lac$^-$}& \multirow{2}{*}{Tc-dep.Lac$^+$}& \multirow{2}{*}{Ratio: $\frac{Lac^-}{\#TcR}$}&\multirow{2}{*}{\parbox[*]{1.2cm}{Genes affected}}\\
				&&&&&&\\
				\midrule
				130 & 2 & 1.5 & 0.5 & 0 & 2/130 & 72\\
				\bottomrule			
			\end{tabular}
		\end{table}
		\section{Diskussion}

	
	\chapter{The Luria-Delbrück fluctuation test}	
		\section{Einleitung}
		\section{Methode}
			\begin{table}[H]
			\centering
			\caption{Verwendete biologische Material für die den Luira-Delbrück Fluktationstest. Die Stämme wurden in E25-Medium (E Salz Minimal Medium mit 0.0025 $\%$ Glukose) kultiviert bevor es im Versuch in E1000 kultiviert wurde.}
			\label{tab: exp4-biologisches Material}
			\begin{tabular}{ccp{7.5cm}}
				\toprule
				Biologisches Material& Stamm & Phänotyp\\
				\midrule
				\multirow{2}{*}{\parbox[t]{2cm}{Salmonella Wildtyp}}  & \multirow{2}{*}{EM7653} & \multirow{2}{*}{MvP103 sseC::aphT (Km$^R$) wildtype}\\
				&&\\
				&&\\
				\multirow{3}{*}{\parbox[t]{2cm}{Salmonella Mutant}} & \multirow{3}{*}{EM8021} &\multirow{3}{*}{MvP103 sseC::aphT (Km$^R$) $\Delta$mutS::FCF (Cm$^R$)} \\
				&&\\
				&&\\
				
				\bottomrule			
			\end{tabular}
		\end{table}
		
		
		\section{Ergebnis}
		
		\section{Diskussion}
	
	\chapter{Phenotypic analysis of type-III protein secretion}
	
		\section{Analysis of protein secretion via the fT3SS in Salmonella mutants using SDS-PAGE and western blot}
	
		\subsection{Einleitung}
		
		\subsection{Methode}
			\begin{table}[H]
				\centering
				\caption{Salmonella-Stämme die für den Versuch der Analyse der Sekretionsprotei mittels SDS-PAGE und Western Blot verwendet wurde.}
				\label{tab: exp6-biologisches Material part1}
				\begin{tabular}{ccp{7.5cm}}
					\toprule
					Biologisches Material& Stamm & Phänotyp\\
					\midrule
					\multirow{2}{*}{\parbox[t]{2cm}{Salmonella Wildtyp}}  & \multirow{2}{*}{EM8016} & \multirow{2}{*}{\parbox[t]{7.5cm}{MvP103 sseC::aphT (Km$^R$) $\Delta$ hin5717::FRT (fliC-ON)}}\\
					&&\\
					&&\\
					\multirow{3}{*}{\parbox[t]{2cm}{Salmonella Mutant}} & \multirow{3}{*}{EM8035} &\multirow{3}{*}{\parbox[t]{7.5cm}{MvP103 sseC::aphT (Km$^R$) $\Delta$hin5717::FRT (fliC-ON) $\Delta$fliP6659::tetRA}} \\
					&&\\
					&&\\
					
					\bottomrule			
				\end{tabular}
			\end{table}
			
			\subsection{Ergebnis}
			
			\subsection{Diskussion}
			
		\section{Luciferase assay to measure gene expression}
		
			\subsection{Einleitung}
			
			\subsection{Methode}
				\begin{table}[H]
				\centering
				\caption{Salmonella-Stämme die für den Luciferase Assay verwendet wurde.}
				\label{tab: exp6-biologisches Material part2}
				\begin{tabular}{ccp{7.5cm}}
					\toprule
					Biologisches Material& Stamm & Phänotyp\\
					\midrule
					\multirow{2}{*}{\parbox[t]{2cm}{Salmonella Wildtyp 1 }}  & \multirow{2}{*}{EM8009} & \multirow{2}{*}{\parbox[t]{7.5cm}{pRG38 (P$_{flhD}$-luxCDABE; Tc$^R$)}}\\
					&&\\
					&&\\
					\multirow{2}{*}{\parbox[t]{2cm}{Salmonella Wildtyp 2}}  & \multirow{2}{*}{EM8008} & \multirow{2}{*}{\parbox[t]{7.5cm}{pRG52 (P$_{flhB}$-luxCDABE; Tc$^R$)}}\\
					&&\\
					&&\\
					\multirow{2}{*}{\parbox[t]{2cm}{Salmonella Wildtyp 3}}  & \multirow{2}{*}{EM8007} & \multirow{2}{*}{\parbox[t]{7.5cm}{pRG19 (P$_{motA}$-luxCDABE; Tc$^R$)}}\\
					&&\\
					&&\\
					\multirow{3}{*}{\parbox[t]{2cm}{Salmonella Mutant 1}} & \multirow{3}{*}{EM8044} &\multirow{3}{*}{\parbox[t]{7.5cm}{$\Delta$flgE22964::FCF / pRG38 (P$_{flhD}$-luxCDABE; Tc$^R$)}} \\
					&&\\
					&&\\
					\multirow{3}{*}{\parbox[t]{2cm}{Salmonella Mutant 2}} & \multirow{3}{*}{EM8045} &\multirow{3}{*}{\parbox[t]{7.5cm}{$\Delta$flgE22964::FCF / pRG52 (P$_{flhB}$-luxCDABE; Tc$^R$)}} \\
					&&\\
					&&\\
					\multirow{3}{*}{\parbox[t]{2cm}{Salmonella Mutant 3}} & \multirow{3}{*}{EM8043} &\multirow{3}{*}{\parbox[t]{7.5cm}{$\Delta$flgE22964::FCF / pRG19 (P$_{flhB}$-luxCDABE; Tc$^R$)}} \\
					&&\\
					&&\\
					
					\bottomrule			
				\end{tabular}
			\end{table}
			
		\section{Motility assay to analyze the swimmng behavor of Salmonella mutants}
		
			\subsection{Einleitung}
			
			\subsection{Methode}
				\begin{table}[H]
				\centering
				\caption{Salmonella-Stämme die für den Motility-Assay verwendet wurde.}
				\label{tab: exp6-biologisches Material part3}
				\begin{tabular}{ccp{5cm}}
					\toprule
					Biologisches Material& Stamm & Phänotyp\\
					\midrule
					\multirow{3}{*}{\parbox[t]{2cm}{Salmonella Wildtyp }}  & \multirow{3}{*}{EM8016} & \multirow{3}{*}{\parbox[t]{4.5cm}{MvP103 sseC::aphT (Km$^R$) $\Delta$hin5717::FRT (fliC-ON)}}\\
					&&\\
					&&\\
					\multirow{3}{*}{\parbox[t]{2cm}{Salmonella Mutant 1}}  & \multirow{3}{*}{EM8033} & \multirow{3}{*}{\parbox[t]{4.5cm}{MvP103 sseC::aphT (Km$^R$) $\Delta$hin5717::FRT (fliC-ON) $\Delta$fliT5758::FCF}}\\
					&&\\
					&&\\
					&&\\
					\multirow{3}{*}{\parbox[t]{2cm}{Salmonella Mutant 2}}  & \multirow{3}{*}{EM8034} & \multirow{3}{*}{\parbox[t]{4.5cm}{MvP103 sseC::aphT (Km$^R$) $\Delta$hin5717::FRT (fliC-ON) $\Delta$fliS5728::FRT}}\\
					&&\\
					&&\\
					&&\\
					\multirow{3}{*}{\parbox[t]{2cm}{Salmonella Mutant 3}} & \multirow{3}{*}{EM8035} &\multirow{3}{*}{\parbox[t]{4.5cm}{MvP103 sseC::aphT (Km$^R$) $\Delta$hin5717::FRT (fliC-ON) $\Delta$fliP6659::tetRA}} \\
					&&\\
					&&\\
					&&\\
					\multirow{3}{*}{\parbox[t]{2cm}{Salmonella Mutant 4}} & \multirow{3}{*}{EM8036} &\multirow{3}{*}{\parbox[t]{4.5cm}{MvP103 sseC::aphT (Km$^R$) $\Delta$hin5717::FRT (fliC-ON) $\Delta$fliC7715::tetRA}} \\
					&&\\
					&&\\
					&&\\
					\bottomrule			
				\end{tabular}
			\end{table}
			
			\subsection{Ergebnis}
			
			\subsection{Diskussion}
		
		
	\chapter{Dissecting the regulatory mechanism of an unknown flagellar regulator}
		\section{Einleitung}
		
		\section{Methode}
			\begin{table}[H]
			\centering
			\caption{Salmonella-Stämme die für die Untersuchung des unbekannten Flaggella Regulator verwendet wurde.}
			\label{tab: exp8-biologisches Material}
			\begin{tabular}{ccp{5cm}}
				\toprule
				Biologisches Material& Stamm & Phänotyp\\
				\midrule
				\multirow{3}{*}{\parbox[t]{2cm}{Salmonella Wildtyp }}  & \multirow{3}{*}{EM8017} & \multirow{3}{*}{\parbox[t]{5cm}{fliL23026::MudJ-Cm (Km$^R$ in MudJ replaced by FCF = Cm$^R$)}}\\
				&&\\
				&&\\
				\multirow{3}{*}{\parbox[t]{2cm}{Salmonella Mutant}}  & \multirow{3}{*}{EM9900} & \multirow{3}{*}{\parbox[t]{5cm}{fliL23026::MudJ-Cm (Km$^R$ in MudJ replaced by FCF = Cm$^R$) $\Delta$regulator::tetRA}}\\
				&&\\
				&&\\
				&&\\
				\bottomrule			
			\end{tabular}
		\end{table}
		
		\section{Ergebnis}
		
		\section{Diskussion}

	
	
	\chapter{Anhang}
\begin{table}[H]
	\centering
	\caption{Der Transposon T-POP wurde in die Recipientenzellen des Stammes EM8052 (Table \ref{tab: exp2-biologisches Material}) über Nacht mit einem Phagenlysat TH3468 (Table \ref{tab: exp2-biologisches Material}) transduziert. Anzahl der Kolonien von Gruppe 1-8, die auf MacConkey-lactose Platten mit Tetracyclin und MacConkey-lactose Platten ohne Tetracyclin gewachsen sind, wurden bestimmt und die Anzahl der betroffene Gene ermittelt.}
	\label{tab: exp2-Rohdaten}
	\begin{tabular}{cccccccc}
		\toprule
		\multirow{2}{*}{Gr.} & \multirow{2}{*}{$\#$TcR} & \multirow{2}{*}{Lac$^-$}&\multirow{2}{*}{lac$^-$ (beide Platten)} & \multirow{2}{*}{Tc-dep.-Lac$^-$}& \multirow{2}{*}{Tc-dep.Lac$^+$}& \multirow{2}{*}{Ratio: $\frac{Lac^-}{\#TcR}$}&\multirow{2}{*}{\parbox[*]{1.2cm}{Genes affected}}\\
		&&&&&&&\\
		\midrule
		1 & 246 & 5 & 3 & 2 & 0 & 5/246 & 81\\
		2 & 77 & 1 & 0 & 0 & 1 & 1/77 & 52\\
		3 & 115 & 1 & 1 & 0 & 0 & 1/115& 35\\
		4 & 93 & 2 & 1 & 1 & 0 & 2/93 & 86\\
		5 & 33 & 1 & 1 & 0 & 0 & 1/33& 121\\
		6 & 95 & 0 & 0 & 0 & 0 & NA & NA\\
		7 & 106 & 1 & 1 & 0 & 0 & 1/106 & 38\\
		8 & 274 & 6 & 5 & 1 & 0 & 6/274 & 88\\
		\bottomrule			
	\end{tabular}
\end{table}
	
	
	
	\addcontentsline{toc}{section}{Bibliography}
	\bibliographystyle{plainurl}
	\nocite{*}
	\bibliography{Literatur}
	\newpage
\end{document}