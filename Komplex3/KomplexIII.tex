\documentclass[10pt,a4paper]{article}
\usepackage[a4paper, left=3cm, right=3cm, top=3cm, bottom=3cm, headsep=10mm, footskip=12mm]{geometry}
\usepackage[T1]{fontenc}
\usepackage[ngerman, english]{babel}    % mehrsprachiger Textsatz
% babel: letzte Sprache in Optionen zeigt die Sprache des Dokumentes
% und kann durch den Befehl \selectlanguage{} geaendert werden
% Passen Sie die Optionen des babel-Paketes nach Bedarf an!
\usepackage{float}
\usepackage{graphicx}
\usepackage{pdflscape}
\usepackage{mathtools}
\usepackage{amssymb, amsmath, amstext}
\usepackage{amsthm}
\usepackage{xcolor}
\usepackage{nameref}
\usepackage{siunitx}
\usepackage{makecell}
\usepackage{hyperref}
\usepackage{enumitem}
\usepackage[superscript,biblabel]{cite}
\usepackage{caption}
\usepackage{subcaption}
\usepackage{tabularx} 			% Tabellen erzeugen
\usepackage{multirow}			 % Zeilen in Tabellenbearbeitung
\usepackage{multicol} 			% Spalten in Tabellenbearbeitung 
\usepackage{lmodern}                        % Ersatz fuer Computer Modern-Schriften 
\usepackage{amsmath}                                           % zum besseren Aussehen am Bildschirm
\usepackage{booktabs} % für schönere Tabellen
\usepackage{sidecap}
\usepackage{rotating} % für die Landscape-Umgebung
\usepackage{afterpage}
\definecolor{Bluetitle}{HTML}{1F3864}
\definecolor{Greyish}{HTML}{5A5A5A}
\renewcommand{\refname}{Reference}
\usepackage{array,multirow}
\newcommand{\specialcell}[2][c]{%
	\begin{tabular}[#1]{@{}c@{}}#2\end{tabular}}




\begin{document}
	
	\begin{titlepage}
		\begin{center}
			\begin{figure}[h!tbp]
				\includegraphics[width=\linewidth]{HUlogo.PNG}
			\end{figure}
			\vspace*{0.5cm}
			
			\textcolor{Bluetitle}{\textbf{\huge V2 Photosensitivitätsanalyse von Grünalgen C. reinhardtii}}\par
			
			\vspace*{1.4cm}
			
			\textcolor{Greyish}{\textbf{Versuchsdurchführende}}\par
			\textcolor{Greyish}{Oscar Moore (634083)}\par
			\textcolor{Greyish}{Frido (....)}\par
			\textcolor{Greyish}{Philipp.. (...)}\par
			\textcolor{Greyish}{Daniel... (...)}\par
			\textcolor{Greyish}{Huyen Anh Nguyen (572309)}\par
			\vspace*{0.5cm}
			\textcolor{Greyish}{\textbf{Versuchsort}}\par
			\textcolor{Greyish}{Campus Nord, Haus 9}\par
			\textcolor{Greyish}{R2002}\par
			\vspace*{0.5cm}
			\textcolor{Greyish}{\textbf{Versuchsbetreuer}}\par
			\textcolor{Greyish}{Prof. Dr. rer. nat. Bernhard Grimm}\par
			
			\vspace*{1.0 cm}
			
			\textcolor{Greyish}{11. Juni 2024}\par
			
			\vspace*{1.0 cm}
			
			
		\end{center}
		
		\tableofcontents
		
	\end{titlepage}
	
	
	\section{Einführung}
	

	
	\section{Material und Methode}
	Für diesen Versuch wurde die Tabakpflanze Nicotiana tabacum als Wildtyp und die antisense FC1 mutierte Variante verwendet.
		\subsection{Herstellung des Pigmentextraktes}
		\subsection{Photometrische Bestimmung Chlorophylle a und b}
		\subsection{DC-Trennung Chlorophylle und Carotinoide}
		Das Pigmentextrakt im basischen Aceton wurde jeweils vom Wildtyp-Extract und Mutanten-Extract 5 mL entnomen und mit 1 mL Petrolether versetzt und 3 mal vorsichtig invertiert.
		Die Proben wurden für 10 Minuten im Eis inkubiert und die obere dunkelgrüne Phase für die weiteren Versuchen entnommen.\\
		\\
		50 Microliter von den beiden Proben wurden auf einer Dünnschichtchromatographi-Platte (agekürzt: DC-Platte) als breite Bande aufgetragen.
		In einem mit Laufmittel (Petrolether/Aceton/Isopropanol/Wasser, 400:80:48:1)-abgesättigte DC-Kammer wird die DC-Platte inkubiert und der Lauf wurde gestoppt, als diese 5 mm Abstand zur DDC-Plattenkante.
		\subsection{Fluoreszenz des Chlorophyllextractes}
		\subsection{Phäophytinbildung}
	
	\section{Ergebnis und Disskusion}


	
\end{document}